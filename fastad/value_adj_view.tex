\subsection{Class ValueAdjView}\label{ssec:value_adj_view}

Since every expression node needs to view both its value and adjoint,
it is useful to create another abstraction that does so.
\verb|ValueAdjView| is a very light wrapper of \verb|ValueView|.
The following is a simplified definition:
\begin{lstlisting}[style=customcpp]
template <class ValueType, class ShapeType>
struct ValueAdjView: ValueView<ValueType, ShapeType>
{
    using ptr_pack_t = util::PtrPack<value_t>;

    ValueAdjView(value_t* val, 
                 value_t* adj,
                 size_t rows=1, 
                 size_t cols=1)
        : base_t(val, rows, cols)
        , adj_view_(adj, rows, cols)
    {}
     
    var_t& get_adj() { return adj_view_.get(); }
    const var_t& get_adj() const { return adj_view_.get(); }

    ptr_pack_t bind(ptr_pack_t begin)
    { 
        begin.val = base_t::bind(begin.val);
        begin.adj = adj_view_.bind(begin.adj);
        return begin;
    }

private:
    base_t adj_view_;
};
\end{lstlisting}
A \verb|ValueAdjView| object \emph{is} a \verb|ValueView| object
and additionally contains another \verb|ValueView| object to view the adjoints.
We expose more member functions such as \verb|get_adj| 
to easily interface with both the underlying value and adjoints.
\verb|bind| is written a little differently to take in a pointer pack of type \verb|ptr_pack_t|.
This pointer pack contains two pointers --- one for the value region and one for the adjoint.
The function then binds the value and adjoint by delegating to the underlying \verb|ValueView| objects.
Each of the pointers are then updated to point to the next unbound memory location and the pack is returned.
