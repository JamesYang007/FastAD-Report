\subsection{Example}\label{ssec:example}

In this section, we show how FastAD can be used to differentiate Eq.~\ref{eq:f-example}.
The following is one example of how this can be done:
\begin{lstlisting}[style=customcpp]
    ad::Var<double, ad::scl> x1, x2, x3;
    auto expr = ad::bind(
        (ad::sin(x1) + ad::cos(x2) * x3 - ad::log(x3))
    );
    double f = ad::autodiff(expr);
    std::cout << f << std::endl;
    std::cout << x1.get_adj() << std::endl; // print df/dx1
    std::cout << x2.get_adj() << std::endl; // print df/dx2
    std::cout << x3.get_adj() << std::endl; // print df/dx3
\end{lstlisting}
The user first declares AD variables to represent the independent variables.
In analogy to Fig.~\ref{fig:expr-tree-example}, the declared \code{xi} variables
correspond to the gray $x_i$ containers in the figure and the \code{xi}
that are used during the expression construction correspond to $w_i$ nodes in the figure.
The second template parameter of \code{Var} describes the general shape of the variable.
\code{scl} represents a scalar.
Then the user creates the expression for the function that they want to differentiate.
The way we construct expressions is very different from most AD libraries.
When the expression is constructed, it does not forward-evaluate.
In fact, no memory for values and adjoints for the intermediate expression nodes is even allocated.
Rather, it returns a cheap, stack-allocated object that stores information
about \emph{how} to evaluate this expression --- this is made possible through expression templates.
The expression is then wrapped with a call to \code{ad::bind},
which will create a wrapper class of the expression object
that does something like the following:
\begin{lstlisting}[style=customcpp]
    std::vector<double> val_buf, adj_buf;
    auto expr = ad::sin(x1) + ad::cos(x2) * x3 - ad::log(x3);
    util::SizePack size_pack = expr.bind_cache_size();
    val_buf.resize(size_pack[0]);
    adj_buf.resize(size_pack[1]);
    expr.bind({val_buf.data(), adj_buf.data()});
\end{lstlisting}
We refer to the memory region allocated for values and adjoints for
expression nodes as the \emph{cache}.
Note that this does not include the value and adjoint region 
for the independent variables and its viewers,
e.g. $x_i$ and $w_i$ in Fig.~\ref{fig:expr-tree-example}.
\code{bind\_cache\_size} is a member implemented by every expression node
and computes the number of value and adjoint elements
needed for the subtree starting at that current node.
It returns a \code{SizePack} object that contains two unsigned long integers
that represent the value and adjoint sizes.
This tells us exactly how much to allocate for the values and adjoints 
for the entire expression tree.
\code{bind\_cache} takes in a pointer pack object,
which can be constructed using an initializer list,
and binds each expression node to view 
a region of the value and adjoint buffer starting from the given pointers.
The binding follows a simple pattern:
the pointer pack contains the first unbound memory location and 
the current node must view a certain number of elements from this starting point.
The node then returns a new pack where the pointers are incremented with however much was taken.
The total amount taken for each node must coincide with the \code{SizePack} object from before.
This process defines our lazy allocation strategy
since we allocate memory only when the full expression is known.

\code{autodiff} is our built-in function that forward and backward-evaluates an expression.
It returns the forward-evaluated (function) value
and the gradient will be stored in the containers \code{xi}.
The user can retrieve these values by using the member function \code{get\_adj}.

The following is another example using \code{VarView},
which, unlike \code{Var}, does not own any storage for its value and adjoint,
but rather views external memory regions for these quantities.
\begin{lstlisting}[style=customcpp]
    std::vector<double> val_buf(3), adj_buf(3);
    ad::VarView<double, ad::scl> x1(val_buf.data(), 
                                    adj_buf.data());
    ad::VarView<double, ad::scl> x2(val_buf.data()+1, 
                                    adj_buf.data()+1);
    ad::VarView<double, ad::scl> x3(val_buf.data()+2, 
                                    adj_buf.data()+2);
    // rest is same...
\end{lstlisting}
Each \code{xi} views the $i$th element of \code{val\_buf} and \code{adj\_buf}.

Finally, although multi-dimensional shapes are supported with the intention of
performing vectorized operations, one can certainly use it in this context as well:
\begin{lstlisting}[style=customcpp]
    ad::Var<double, ad::vec> x(3); // size is 3
    auto expr = ad::bind(
        (ad::sin(x[0]) + ad::cos(x[1]) * x[2] - ad::log(x[2]))
    );
    double f = ad::autodiff(expr);
    std::cout << f << std::endl;
    std::cout << x.get_adj() << std::endl; // print full gradient
\end{lstlisting}
Note that \code{x} is declared with a vector-shape (\code{vec}).
The size must be known at the time of construction.
We provide a subscript-operator for vector-shapes
that makes it convenient to construct the same expression as before.
The adjoint of \code{x} is the full gradient and 
is represented as an \code{Eigen} matrix-like object,
hence the user can easily interface with this adjoint by using \code{Eigen} API.\@
For example, \code{Eigen} matrix objects can be passed to \code{std::cout}
to print out all of the elements.
