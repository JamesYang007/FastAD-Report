\section{Tips}
\frame{\tableofcontents[currentsection]}

\begin{frame}{Reverse-Mode}
\begin{itemize}
    \item Extract common sub-expression into a placeholder variable, 
    to avoid recomputation, e.g.
    \begin{align*}
        t &= x + y \\
        f(t) &= \sin(t) \cos(t) 
    \end{align*}
    \item Use package-provided functions mostly to take advantage of vectorization, e.g.
    \begin{align*}
        \text{for-loop sum over } x \, &\text{ \xmark} \\
        \text{sum}(x) \, &\text{ \cmark} 
    \end{align*}
\end{itemize}
\end{frame}

\begin{frame}{Reverse-Mode}
\begin{itemize}
    \item Minimize number of nodes in computation graph,
    e.g. if $x \in \R^{10}$,
    \begin{align*}
        x[1] + \ldots + x[10] \implies 9 \, \text{ adjoints} \; &\text{ \xmark} \\
        \text{sum}(x) \; \implies 1 \, \text{ adjoint} \; &\text{ \cmark}
    \end{align*}
    \item Minimize size of each node of computation graph,
    e.g. if $x, y \in \R^n$,
    \begin{align*}
        \text{sum}(x+y) \implies O(n) \, \text{memory} \; &\text{ \xmark} \\
        \text{sum}(x) + \text{sum}(y) \implies O(1) \, \text{memory} \; &\text{ \cmark}
    \end{align*}
\end{itemize}
\end{frame}