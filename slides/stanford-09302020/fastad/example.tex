\subsection{Example}
\frame{\tableofcontents[currentsubsection]}

\begin{frame}[fragile]
\frametitle{Example Code}
\begin{lstlisting}[style=customcpp]
    ad::Var<double, ad::vec> x(3); // size is 3
    auto expr = ad::bind((
        ad::sin(x[0]) 
        + ad::cos(x[1]) * x[2] 
        - ad::log(x[2])
    ));

    double f = ad::autodiff(expr);

    std::cout << f << std::endl;
    std::cout << x.get_adj() << std::endl;
\end{lstlisting}

\begin{itemize}
    
\item \code{ad::bind} wraps the lazy allocation strategy.

\item \code{autodiff} is the functional that forward and backward-evaluates.

\end{itemize}
\end{frame}

\begin{frame}[fragile]
\frametitle{Example Code: Complex}
\begin{lstlisting}[style=customcpp]
    ad::Var<double, ad::mat> X(10, 3);
    ad::Var<double, ad::vec> w(3);
    ad::Var<double, ad::scl> s1, s2;

    auto expr = ad::bind((
        s1 = ad::sum(w),
        s2 = ad::sum(X),
        s1*ad::dot(X, w) + ad::sin(s1*s2)*norm(X)
    ));

    double f = ad::autodiff(expr);
\end{lstlisting}

\begin{itemize}
    
\item Overloaded \code{operator,} to store sequence of expressions at compile-time.

\end{itemize}
\end{frame}
