\section{Benchmarks}\label{sec:benchmark}

In this section, we compare performances of 
various libraries against FastAD for a range of examples\footnotemark.
\footnotetext{github page: https://github.com/JamesYang007/ADBenchmark}
The following is an alphabetical list of the libraries used for benchmarking:\@
\begin{itemize}
    \item \href{http://www.met.reading.ac.uk/clouds/adept/}{Adept 2.0.8}~\cite{hogan:2014}
    \item \href{https://github.com/coin-or/ADOL-C}{ADOL-C 2.7.2}~\cite{griewank:1996}
    \item \href{https://coin-or.github.io/CppAD/doc/cppad.htm}{CppAD 20200000}~\cite{bell:2020}
    \item \href{https://github.com/JamesYang007/FastAD}{FastAD 3.1.0}
    \item \href{https://github.com/trilinos/Trilinos/tree/master/packages/sacado}{Sacado 13.0.0}~\cite{phipps:2009}
    \item \href{https://github.com/stan-dev/math}{Stan Math Library 3.3.0}~\cite{carpenter:2015}
\end{itemize}
All the libraries are at their most recent release at the time of benchmarking.
These libraries have also been used by others~\cite{carpenter:2015}\cite{margossian:2018}\cite{hogan:2014}.

All benchmarks were run on a Google Cloud Virtual Machine with the following configuration:
\begin{itemize}
    \item \textbf{OS}: Ubuntu 18.04.1 
    \item \textbf{Architecture}: x86 64-bit
    \item \textbf{Processor}: Intel Xeon Processor 
    \item \textbf{Cores}: 8
    \item \textbf{Compiler}: GCC10
    \item \textbf{C++ Standard}: 17
    \item \textbf{Compiler Optimization Flags}: \code{-O3 -march=native} 
\end{itemize}

All benchmarks benchmark the case where a user wishes to differentiate
a scalar function $f$ for different values of $x$.
This is a very common use-case.
For example, in the Newton-Raphson method,
we have to compute $f'(x_n)$ at every iteration with the updated $x_n$ value.
In HMC and NUTS, the leapfrog algorithm frequently
updates a quantity called the ``momentum vector'' $\rho$ 
with $\nabla_\theta \log(p(\theta, x))$ ($x$ here is treated as a constant),
where $\theta$ is a ``position vector'' that also gets frequently updated.

Our benchmark drivers are very similar to the ones used by Stan~\cite{carpenter:2015}.
As noted in~\cite{margossian:2018}, there is no standard set of benchmark examples for AD,
but since Stan is most similar to FastAD in both design and intended usage,
we wanted to keep the benchmark as similar as possible.

We measure the average time to differentiate a function
with an initial input and 
save the function evaluation result as well as the gradient.
After timing each library, the gradients are compared with manually-written gradient computation to check accuracy.
All libraries had some numerical issues for some of the examples,
but the maximum proportion of error to the actual gradient values was on the order of $ 10^{-15}$, which is negligible.
Hence, in terms of accuracy, all libraries were equally acceptable.
Finally, all benchmarks were written in the most optimal way for every library based on their documentation.

Every benchmark also times the ``baseline'', 
which is a manually-written forward evaluation (computing function value).
This will serve as a way to measure the extra overhead of computing the gradient relative to computing the function value.
This approach was also used in~\cite{carpenter:2015},
however in our benchmark, the baseline is also optimized to be vectorized.
Hence, our results for all libraries with respect to the baseline may differ from those in the reference.

Sections~\ref{ssec:sum_prod}-\ref{ssec:normal_log_pdf} 
cover some micro-benchmarks where we benchmark commonly used functions: 
summation and product of elements in a vector, 
log-sum-exponential, 
matrix multiplication, 
and normal log-pdf.
Sections~\ref{ssec:regression}-\ref{ssec:stochastic_volatility} 
cover some macro-benchmarks where we benchmark practical examples: 
a regression model and a stochastic volatility model.

\subsection{Sum and Product}

\begin{figure*}[t]
    \centering
    \begin{subfigure}[b]{0.475\textwidth}
        \centering
        \includegraphics[width=\textwidth]{figs/sum_fig.png}
        \caption{Sum}\label{fig:sum}
    \end{subfigure}
    \hfill
    \begin{subfigure}[b]{0.475\textwidth}
        \centering
        \includegraphics[width=\textwidth]{figs/sum_iter_fig.png}
        \caption{Sum (naive, for-loop-based)}\label{fig:sum_iter}
    \end{subfigure}
    \vskip\baselineskip%
    \begin{subfigure}[b]{0.475\textwidth}
        \centering
        \includegraphics[width=\textwidth]{figs/prod_fig.png}
        \caption{Product}\label{fig:prod}
    \end{subfigure}
    \hfill
    \begin{subfigure}[b]{0.475\textwidth}
        \centering
        \includegraphics[width=\textwidth]{figs/prod_iter_fig.png}
        \caption{Product (naive, for-loop-based)}\label{fig:prod_iter}
    \end{subfigure}
    \caption{%
        Sum and product benchmarks of other libraries against FastAD 
        plotted relative to FastAD average time.
        Fig.~\ref{fig:sum},\ref{fig:prod} use built-in functions whenever available.
        Fig.~\ref{fig:sum_iter},\ref{fig:prod_iter} use the naive iterative-based code for all libraries.
    }\label{fig:sum_prod}
\end{figure*}

The summation function is defined as $f_s:\R^n \to \R$ where~$f_s(x) = \sum\limits_{i=1}^n x_i$,
and the product function is defined as $f_p:\R^n \to \R$ where~$f_p(x) = \prod\limits_{i=1}^n x_i$.
The only libraries supporting a built-in function for summation are Adept, FastAD, and Stan;
those that support product are Adept and FastAD.\@
All other libraries use the \verb|Eigen| API:
\begin{lstlisting}[style=customcpp]
    template <class T>
    T operator()(const Eigen::Matrix<T, Eigen::Dynamic, 1>& x) const
    { return x.sum(); }
\end{lstlisting}
As another test, we forced all libraries to use the manually-written for-loop-based summation and product.
Fig.\ref{fig:sum_prod} shows the benchmark results.

In all four cases, it is clear that FastAD outperforms all libraries for all values of $N$
by a significant factor.
For \verb|sum|, Fig.\ref{fig:sum} shows that, asymptotically, 
the next three fastest libraries are Adept, Stan, and Sacado, respectively.
The trend stabilizes starting from $N=2^6=64$ where Adept is about $ 12$ times slower than
FastAD, Stan about $ 18$ times, and Sacado about $ 74$ times.
The naive, for-loop-based benchmark shown in Fig.\ref{fig:sum_iter} exhibits a similar pattern,
although the performance difference with FastAD is less extreme:
Adept is about $ 4$ times slower, 
Stan about $ 6$ times,
and Sacado about $ 6.5$ times.
Nonetheless, this is still a very notable difference.
Although the difference between FastAD and the \verb|double| baseline is much larger in
the \verb|sum| case (2.56 times slower than baseline) than 
\verb|sum_iter| (1.76 times slower than baseline), 
this does not mean that \verb|sum| was slower.
In fact, in terms of absolute time, \verb|sum| was $ 5$ times faster than \verb|sum_iter|. 
Rather this means that the backward-evaluation in the case for \verb|sum| was
as expensive as the forward-evaluation, but in the case for \verb|sum_iter|,
backward-evaluation was much cheaper relative to the forward-evaluation.
This makes sense because \verb|sum| is vectorized and hence takes much shorter time to evaluate
than \verb|sum_iter|, and in both cases the backward-evaluation simply sets the partial derivatives to $ 1$.

For the \verb|prod| and \verb|prod_iter| cases, 
Fig.\ref{fig:prod},\ref{fig:prod_iter} again show
a similar trend as \verb|sum| and \verb|sum_iter|.
Overall, the comparison with FastAD is less extreme.
For \verb|prod|, Adept is about $ 5.6$ times slower than FastAD,
Stan about $ 7.3$ times slower,
and Sacado about $ 16.4$ times slower.
For \verb|prod_iter|, Adept is about $2.9$ times slower,
Stan about $4$ times slower,
and Sacado about $4.6$ times slower.


\subsection{Log-Sum-Exp}

\begin{figure*}[t]
    \centering
    \includegraphics[width=\textwidth]{figs/log_sum_exp_fig.png}
    \caption{%
        Log-sum-exp benchmark of other libraries against FastAD 
        plotted relative to FastAD average time.
    }\label{fig:log_sum_exp}
\end{figure*}

The log-sum-exp function is defined as $f:\R^n \to \R$ 
where~$f(x) = \log(\sum\limits_{i=1}^n e^{x_i})$,
The only library supporting a built-in function for log-sum-exp is Stan.
Fig.\ref{fig:log_sum_exp} shows the benchmark results.

FastAD outperforms all libraries for all values of $N$.
The next three fastest libraries are Adept, Stan, and Sacado, respectively.
The trend stabilizes starting from $N=2^6=64$ where 
Adept is about $ 3$ times slower than FastAD, 
Stan about $ 5$ times, and 
Sacado about $ 17$ times.
Note that FastAD is only marginally slower than the baseline
despite calls to expensive functions like \code{log} and \code{exp}.
In fact, FastAD is only $ 1.5$ times slower than the baseline, 
meaning there is only about $ 50\%$
overhead from one forward-evaluation to also compute the gradient.
This is not surprising since FastAD reuses
the forward-evaluated result for \code{exp} expression
during the backward-evaluation (see Section~\ref{ssec:unary}).


\subsection{Matrix Multiplication}\label{ssec:matrix_mult}

\begin{figure*}[t]
    \centering
    \includegraphics[width=0.8\textwidth]{figs/matrix_product_fig.png}
    \caption{%
        Matrix multiplication benchmark of other libraries against FastAD 
        plotted relative to FastAD average time.
    }\label{fig:matrix_mult}
\end{figure*}

For this benchmark, all matrices are square matrices of the same size.
In order to have a scalar target function,
we add another step of adding all of the entries of the matrix multiplication, i.e.
\[
    f(A, B) = \sum\limits_{i=1}^{K} \sum\limits_{j=1}^{K} {(A \cdot B)}_{ij}
\]
Fig.~\ref{fig:matrix_mult} shows the benchmark results.

FastAD is still the fastest library for all values of $N$, 
but Stan performs much closer to FastAD than in the previous examples.
All other libraries consistently take longer than both FastAD and Stan as $N$ increases.
Towards the end at around $N=2^{14}$, 
Stan is about $ 2$ times slower.
For moderate sized $N \in [2^{8}, 2^{10}]$, Stan is about $ 3$ times slower.

This example really highlights the benefits of vectorization.
As noted in Section~\ref{ssec:vectorization},
this was the one benchmark example where Stan was able to produce vectorized code,
which is consistent with Figure~\ref{fig:matrix_mult} 
that Stan is the only library that has the same 
order of magnitude as FastAD.
Other libraries did not produce vectorized code.

The comparison with the baseline shows that FastAD takes $ 3.27$ times longer.
Note that forward-evaluation requires one matrix multiplication between two $K\times K$ matrices,
and backward-evaluation additionally requires two matrix multiplications of the same order,
one for each adjoint:
\begin{align*}
    \frac{\partial f}{\partial A} 
    &= \frac{\partial f}{\partial (A\cdot B)} \cdot B^T, \,
    \frac{\partial f}{\partial B} 
    = A^T \cdot \frac{\partial f}{\partial (A\cdot B)}
\end{align*}
Hence, in total, one AD evaluation requires three matrix multiplications between two $K\times K$ matrices.
If we approximate a manually-written gradient computation to take 
three times as long as the baseline (one multiplication), 
FastAD time relative to this approximated time
is $\frac{3.27}{3} = 1.09$.
This shows then that FastAD only has about $ 9\%$ overhead 
from a manually-written code, which is extremely optimal.


\subsection{Normal Log-PDF}\label{ssec:normal_log_pdf}

\begin{figure*}[t]
    \centering
    \includegraphics[width=0.7\textwidth]{figs/normal_log_pdf_fig.png}
    \caption{%
        Normal log-pdf benchmark of other libraries against FastAD 
        plotted relative to FastAD average time.
    }\label{fig:normal_log_pdf}
\end{figure*}

The normal log-pdf function is defined up to a constant as:
\[
    f(x) = -\frac{1}{2\sigma^2} \sum\limits_{i=1}^N \paren{x_i - \mu}^2 
           -N\log(\sigma)
\]
For this benchmark, $\mu = -0.56,\,\sigma = 1.37$ and are kept as constants.
Fig.~\ref{fig:normal_log_pdf} shows the benchmark results.

FastAD is the fastest library for all values of $N$.
The trend stabilizes at around $N=2^{7}=128$.
Towards the end, Adept is about $ 9$ times slower,
and Stan about $ 19$ times slower.
Comparing all libraries,
the overall difference we see in this example is the largest we have seen so far,
and this is partly due to how we compute $\log(\sigma)$.
Section~\ref{ssec:compile-time-opt} showed that we can check at compile-time
whether a certain variable is a constant, in which case,
we can perform a more optimized routine.
In this case, since $\sigma$ is a constant, it computes the normalizing constant
$\log(\sigma)$ once and gets reused over multiple AD evaluations 
with no additional cost during runtime,
which saves a lot of time since logarithm is a relatively expensive function.


\subsection{Bayesian Linear Regression}\label{ssec:regression}

\begin{figure*}[t]
    \centering
    \includegraphics[width=\textwidth]{figs/regression_fig.png}
    \caption{%
        Bayesian linear regression benchmark of Stan against FastAD 
        plotted relative to FastAD average time.
    }\label{fig:regression}
\end{figure*}

This section marks the first macro-benchmark example.
We consider the following Bayesian linear regression model:
\begin{align*}
    y &\sim N\paren{X\cdot w + b, \sigma^2} \\
    w &\sim N\paren{0,1} \\
    b &\sim N\paren{0,1} \\
    \sigma &\sim Unif\paren{0.1, 10.}
\end{align*}
The target function is the log of the joint probability density function (up to a constant).
For this benchmark, we only consider Stan since they specialize in differentiating such functions
and because the other libraries were not well-suited to implement this efficiently.
The fill function for this functor will resize a vector of size $N = 2^K$ as $\tilde{N} = (K + 1) \cdot 10 + 2$,
where the first $(K+1) \cdot 10$ values refer to $w$ and the rest refer to $b$ and $\sigma$, respectively.
It also resizes its private members $y \in \R^{1000}$ and $X \in \R^{1000 \times \tilde{N}}$, which are constants.
All quantities are randomly generated uniformly in $(-1,1)$ range,
but $\sigma$ modified to be strictly positive.

FastAD outperforms Stan by 8 times for the largest $N$.
The trend stabilizes starting from around $N=70$.
It is interesting to see that FastAD is only 2 times slower than the baseline,
despite the model consisting of a large matrix multiplication and many normal log-pdf functions.
One of the reasons is that the compiler was able to optimize-out backward-evaluation for $X$, 
since constants implement a no-op for backward-evaluation.
If we assume that the most expensive operation is the matrix multiplication,
AD evaluation approximately takes 2 matrix multiplication between a matrix and a vector.
We can then approximate a lower bound for manually-written gradient computation to be two times the baseline.
The relative time of FastAD to this approximation is
$1.018$, implying less than $ 1.8\%$ overhead from a manually-written code.


\subsection{Stochastic Volatility}\label{ssec:stochastic_volatility}

\begin{figure*}[t]
    \centering
    \includegraphics[width=\textwidth]{figs/stochastic_volatility_fig.png}
    \caption{%
        Stochastic volatility benchmark of Stan against FastAD 
        plotted relative to FastAD average time.
    }\label{fig:stochastic_volatility}
\end{figure*}

This section marks the second and last macro-benchmark example.
We consider the following stochastic volatility model 
taken from \todo{how to cite Stan documentation?}:
\begin{align*}
    y &\sim N(0, e^{h}) \\
    h_{std} &\sim N(0, 1) \\
    \sigma &\sim Cauchy(0,5) \\
    \mu &\sim Cauchy(0,10) \\
    \phi &\sim Unif(-1, 1) \\
    h &= h_{std} \cdot \sigma \\
    h[0] &= \frac{h[0]}{\sqrt{1 - \phi^2}} \\
    h &= h + \mu \\
    h[i] &= \phi \cdot (h[i-1] - \mu),\, i > 0
\end{align*}
The target function is the log of the joint probability density function (up to a constant)
and we wish to differentiate it with respect to $h_{std}, h, \phi, \sigma, \mu$.
For this benchmark, we only consider Stan for the same reasons described in Section~\ref{ssec:regression}.
The fill function for this functor will resize a vector of size $N$ as $\tilde{N} = N + 3$,
where the first $N/2$ values refer to $h_{std}$,
the next $N/2$ values refer to $h$,
and the next three refer to $\phi, \sigma, \mu$, respectively.
We take care of the edge case where $N < 2$ to let $N = 2$ and then carry out the steps above.
Here, $y$ is a constant.
All quantities are random generated uniformly in $(-1,1)$ range,
but $\sigma$ is replaced with its absolute value plus $0.1$ to make it strictly positive.

The following is the Stan functor overload:
\begin{lstlisting}[style=customcpp]
    using namespace stan::math;
    using vec_t = Eigen::Matrix<var, Eigen::Dynamic, 1>;
    size_t N = (x.size() - 3) / 2;
    Eigen::Map<vec_t> h_std(x.data(), N);
    Eigen::Map<vec_t> h(x.data() + N, N);
    auto& phi = x(2*N);
    auto& sigma = x(2*N + 1);
    auto& mu = x(2*N + 2);
    h = h_std * sigma;
    h[0] /= sqrt(1. - phi * phi);
    h += mu * Eigen::VectorXd::Ones(N);
    for (size_t i = 1; i < N; ++i) {
        h[i] += phi * (h[i-1] - mu);
    }
    auto lp = normal_lpdf(y, 0., exp(h / 2.)) +
            normal_lpdf(h_std, 0., 1.) +
            cauchy_lpdf(sigma, 0., 5.) +
            cauchy_lpdf(mu, 0., 10.) +
            uniform_lpdf(phi, -1., 1.);
\end{lstlisting}
The following is the FastAD functor overload:
\begin{lstlisting}[style=customcpp]
    size_t N = (x.size() - 3) / 2;
    ad::VarView<T, ad::vec> h_std(x.data(), x.data_adj(), N);
    ad::VarView<T, ad::vec> h(x.data() + N, x.data_adj() + N, N);
    ad::VarView<T> phi(x.data() + 2*N, x.data_adj() + 2*N);
    ad::VarView<T> sigma(x.data() + 2*N + 1, x.data_adj() + 2*N + 1);
    ad::VarView<T> mu(x.data() + 2*N + 2, x.data_adj() + 2*N + 2);
    auto tp = (
        h = h_std * sigma,
        h[0] /= ad::sqrt(1. - phi * phi),
        h += mu,
        ad::for_each(counting_iterator<>(1),
                     counting_iterator<>(N),
                     [&](size_t i) { return h[i] += phi * (h[i-1] - mu); })
    );
    auto lp = ad::normal_adj_log_pdf(y, 0., ad::exp(h / 2.)) 
            + ad::normal_adj_log_pdf(h_std, 0., 1.)
            + ad::cauchy_adj_log_pdf(sigma, 0., 5.)
            + ad::cauchy_adj_log_pdf(mu, 0., 10.) 
            + ad::uniform_adj_log_pdf(phi, -1., 1.);
    return (tp, lp);
\end{lstlisting}
Fig.\ref{fig:stochastic_volatility} shows the benchmark results.

FastAD outperforms Stan by 3 times for the largest $N$.
The trend seems quite stabilized from the start.
It is interesting to see that FastAD is only marginally slower than the \verb|double| baseline
for moderate to large $N$ values.
The ratio between FastAD time to baseline time is $ 1.07$, which means there is about a $7\%$ overhead only
from just one forward-evaluation to also compute the gradient.
Hence, gradient computation is almost free from just computing the log-pdf,
which puts FastAD at near-optimal performance.

