\section{Benchmarks}

In this section, we compare performances of various libraries against FastAD for a range of examples.
The following is an alphabetical list of the libraries used for benchmarking:\@
\begin{itemize}
    \item \href{http://www.met.reading.ac.uk/clouds/adept/}{Adept 2.0.8}
    \item \href{https://github.com/coin-or/ADOL-C}{ADOL-C 2.7.2}
    \item \href{https://coin-or.github.io/CppAD/doc/cppad.htm}{CppAD 20200000}
    \item \href{https://github.com/JamesYang007/FastAD}{FastAD 3.1.0}
    \item \href{https://github.com/trilinos/Trilinos/tree/master/packages/sacado}{Sacado 13.0.0}
    \item \href{https://github.com/stan-dev/math}{Stan Math Library 3.3.0}
\end{itemize}
All of the libraries are at their most recent release.
Note that these libraries have also been used by others for 
benchmarking~\cite{carpenter:2015}\cite{margossian:2018}\cite{hogan:2014}.

The benchmark configuration is as follows:
\begin{itemize}
    \item \textbf{OS}: MacOSX Catalina Version 10.15.6 
    \item \textbf{Architecture}: x86 64-bit
    \item \textbf{Processor}: 3.4 GHz Quad-Core Intel Core i5
    \item \textbf{Compiler}: GCC10
    \item \textbf{C++ Standard}: 17
    \item \textbf{Compiler Flags}: 
\begin{verbatim}
    -O3 -march=native 
    -DNDEBUG 
    -DEIGEN_NO_DEBUG 
    -DADEPT_STACK_THREAD_UNSAFE 
    -D_REENTRANT 
    -DEIGEN_MATRIXBASE_PLUGIN="stan/math/prim/eigen_plugins.h" 
    -DEIGEN_ARRAYBASE_PLUGIN="stan/math/prim/eigen_plugins.h"
\end{verbatim}
    \item \textbf{FastAD Dependency}: \verb|Eigen3.3.7|
\end{itemize}
While we used the newest GCC compiler, there is nothing in FastAD that relies on GCC10 specifically.
FastAD, however, requires the C++17 standard.
The \verb|-O3| indicates maximum level of compiler optimization,
and \verb|-march=native| exposes all of the available instruction sets 
such that the compiler can choose the best one.
The preprocessor definitions are to disable debugging
for best performance for all libraries, and some definitions were required by the libraries in order to compile.
Finally, FastAD has a dependency on \verb|Eigen| matrix library,
but we note that this is also a dependency for Stan.
The only application running was \verb|iTerm2| (terminal),
the benchmark program, and system-related applications.

All benchmarks benchmark the case where a user wishes to differentiate
a scalar function $f$ for different values of $x$.
This is a very common use-case.
For example, in the Newton-Raphson method,
we have to compute $f'(x_n)$ at every iteration with the updated $x_n$ value.
In HMC and NUTS, the leapfrog algorithm frequently
updates a quantity called the ``momentum vector'' $\rho$ 
with $\nabla_\theta \log(p(\theta, x))$ ($x$ here is treated as a constant),
where $\theta$ is a ``position vector'' that also gets frequently updated.

Our benchmark drivers are very similar to the one presented by Carpenter~\cite{carpenter:2015}.
In fact, we mainly took the same benchmark drivers, abstracted some details, and added a few more tests.
Every function to differentiate is defined as a functor and receives a single input,
which is a vector-like object of AD variables representing $x$.
For example, the \verb|sum| functor is as follows:
\begin{lstlisting}[style=customcpp]
struct SumFunc
{
    template <class T>
    T operator()(const Eigen::Matrix<T, Eigen::Dynamic, 1>& x) const
    { return x.sum(); }

    adept::aReal operator()(const adept::aVector& x) const
    { return adept::sum(x); }

    auto operator()(const Eigen::Matrix<stan::math::var, Eigen::Dynamic, 1>& x) const
    { return stan::math::sum(x); }

    template <class T>
    auto operator()(ad::VarView<T, ad::vec>& x) const
    { return ad::sum(x); }

    void derivative(const Eigen::VectorXd& x,
                    Eigen::VectorXd& grad) const
    { grad = Eigen::VectorXd::Ones(x.size()); }

    std::string name() const { return "sum"; }

    void fill(Eigen::VectorXd& x) {
        for (int i = 0; i < x.size(); ++i) {
            x(i) = i;
        }
    }
};
\end{lstlisting}
All functors overload \verb|operator()| to take in library-specific vector of AD variables.
Adept2.0 supports their own array feature and encourages the use of \verb|aVector|.
FastAD supports the AD variable to have a vector-like shape by specifying the second template parameter as \verb|ad::vec|.
All other libraries support a direct interface with~\verb|Eigen::Matrix<AD variable type>|.
Some libraries provide built-in functions to evaluate the functor, and so we specialize whenever that is the case
to call these functions instead of performing the default (first) overload.
The default overload will always try to use \verb|Eigen| API instead of manually written for-loops, as shown above.
However, if a function cannot be easily written with \verb|Eigen| API, then a manually written code is used.
Note that the default overload for a vector of doubles will compute only the forward-evaluation,
and this will serve as a baseline for all benchmarks.
This approach was also used in~\cite{carpenter:2015},
however in our benchmark, this overload for doubles is optimized to be vectorized,
whereas the reference uses a manual for-loop that may not be.
Hence, our results for all libraries with respect to the baseline may differ from those of the reference.
\verb|name| returns an \verb|std::string| object with a short name of the functor
and \verb|fill| fills the initial values prior to starting any benchmarks. 

For brevity, the following is only a snippet of our main
benchmark function~\verb|time_gradients| that benchmarks Adept:
\begin{lstlisting}[style=customcpp]
    auto& adept_pack = packs.at(adb::TestName::adept);
    if (adept_pack.run) {
        adept::aVector x_ad(x.size());
        sw.start();
        for (int i = 0; i < adept_pack.n_iter; ++i) {
            adept_gradient(f, x, x_ad, fx, grad_fx);
        }
        sw.stop();
        check_gradient(grad_fx, expected, adept_pack.name);
        adept_pack.time = sw.elapsed() / adept_pack.n_iter;
    }
\end{lstlisting}
Other libraries are benchmarked in a similar way with \verb|adept| replaced with the other library names.
\verb|packs| is an \verb|std::unordered_map| mapping \verb|TestName| enum value to a \verb|TestPack| object
which contains information such as name of library, number of iterations to run, whether to run or not, etc.
This abstraction was only done to ease the programming and has no bearing on the actual benchmark results.
The caller guarantees to populate \verb|packs| with information for all libraries.
The first line grabs the \verb|TestPack| object associated with Adept and checks if it should be run.
If so, it allocates an (active) Adept vector, which represents the AD variables.
The object \verb|sw| is previously declared (not shown) as \verb|StopWatch<>| and it is used as a stopwatch.
\verb|StopWatch<>| is our own class that wraps \verb|std::chrono::steady_clock| to measure interval time.
The member function \verb|start()| saves the starting time point and \verb|end()| saves the ending time point.
Member function \verb|elapsed()| returns the time elapsed in seconds between \verb|begin()| and \verb|end()| calls.
The elapsed time is computed in nanoseconds and then scaled by $10^{-9}$ to convert into seconds.
We measure the time it takes to differentiate \verb|f|, \verb|adept_pack.n_iter| times,
with the input \verb|x| and the AD variable \verb|x_ad|,
store the function result into \verb|fx|, and save the gradient in \verb|grad_fx|.
For all benchmark examples, \verb|n_iter| is actually the same for all libraries and is set to 10000,
but for matrix multiplication, it is set to 1000 
since the complexity of the function was large enough to reach stable results with fewer iterations.
\verb|f| is the functor representing the function to differentiate,
which was previously described in detail.
The function \verb|check_gradient| is for our debugging purposes to check accuracy.
Note that it is only called after the stopwatch finishes, so it does not affect the benchmark results.
All libraries had some numerical issues for some of the examples,
but the maximum proportion of error to the actual gradient values was on the order of $ 10^{-15}$, which is negligible.
Hence, in terms of accuracy, all libraries were equally acceptable.
Finally, we save the average time into \verb|adept_pack.time|.

Because FastAD works very differently from all of these libraries,
the benchmark setup is quite different:
\begin{lstlisting}[style=customcpp]
    auto& fastad_pack = packs.at(adb::TestName::fastad);
    if (fastad_pack.run) {
        Eigen::VectorXd val_buf;
        Eigen::VectorXd adj_buf;
        ad::VarView<double, ad::vec> x_ad(x.data(),
                                          grad_fx.data(),
                                          x.size());
        auto expr = f(x_ad);
        auto size_pack = expr.bind_cache_size();
        val_buf.resize(size_pack(0));
        adj_buf.resize(size_pack(1));
        expr.bind_cache({val_buf.data(), adj_buf.data()});
        sw.start();
        for (int i = 0; i < fastad_pack.n_iter; ++i) {
            fastad_gradient(expr, fx, grad_fx);
        }
        sw.stop();
        check_gradient(grad_fx, expected, fastad_pack.name);
        fastad_pack.time = sw.elapsed() / fastad_pack.n_iter;
    }
\end{lstlisting}
Unlike all other libraries, calling the functor with \verb|x_ad| will not forward-evaluate.
Rather, it returns a small-sized, stack-allocated expression tree 
(usually only a couple hundred bytes for complicated examples and only tens of bytes for moderate examples).
The analogous structure would be the ``tape'' for ADOL-C and CppAD,
or the stack structure in Adept and Stan.
As mentioned in section~\todo{the one about lazy allocation},
FastAD employs lazy allocation, and so once the expression tree has been created,
we can one-time lazily allocate the \emph{exact} number of bytes for the values and adjoints
associated with the expression nodes, and reuse this memory when evaluating the same expression multiple times.
We then bind the expression object to view this region of memory.
All of these operations are very cheap and fast.
No data is copied and there are only pointer assignments.
The rest of the logic remains the same.

We defined a gradient-computing function such as \verb|adept_gradient| for every library except for Stan,
since they already provided their own.
The code is exactly the same as the one used by Carpenter~\cite{carpenter:2015},
except that the allocations for the AD variables have been moved to before starting the stopwatch as shown above for Adept.
So, we do not show the code for these functions and 
direct the reader to the reference and \todo{our github page}.
As for FastAD, the gradient function is extremely simple:
\begin{lstlisting}[style=customcpp]
    template <class ExprType>
    inline void fastad_gradient(ExprType& expr,
                                double& fx,
                                Eigen::VectorXd& grad_fx) 
    {
        grad_fx.setZero();
        fx = ad::autodiff(expr);
    }
\end{lstlisting}
It first sets the gradient to zero, which is required if multiple AD computations 
are carried out for the same expression viewing that gradient region.
Then \verb|ad::autodiff| both forward and backward-evaluates \verb|expr|
and stores the function value to \verb|fx|.
The backward-evaluation will update \verb|grad_fx| with the gradient values.

The main driver shown below is called \verb|run_test|, which calls
\verb|time_gradients| with various input sizes $N$.
\begin{lstlisting}[style=customcpp]
template <class F>
inline void run_test(F f, 
                     std::unordered_map<adb::TestName, adb::TestPack>& packs,
                     size_t max = 16 * 1024) 
{
    adept::Stack stack;
    std::string file_name = f.name() + "_eval.csv";
    std::fstream fs(file_name, std::fstream::out);
    print_results_header(fs, packs);
    for (int N = 1; N <= max; N *= 2) {
        std::cout << "N=" << N << std::endl;
        Eigen::VectorXd x(N);
        f.fill(x);
        time_gradients(f, x, packs, fs);
    }
}
\end{lstlisting}
The input vector size $N$ grows exponentially in powers of $ 2$ starting from $ 1$ to
$2^{14}$, and for each $N$ we fill the input vector \verb|x| using the functor and call \verb|time_gradients|.
This range of $N$ was enough to determine the asymptotic trend.

To the best of our abilities, we chose the most optimal code for each library 
that would be best suited for this particular scenario.
For example, Adept requires the user to define a stack object that saves the 
expression graph as well as the temporary values and adjoints for each node.
However, this stack object can be reused over many iterations.
Since a real use-case of Adept would likely reuse this stack object, we made this optimization.

For simplicity, we do not actually provide new $x$ values, but the benchmark code
is written as if $x$ may be different at every iteration; that is,
no optimizations were made for any libraries to exploit the fact that 
the $x$ values are actually kept the same.
This is important because most of the libraries require an initial step of creating AD variables
and copying the input values into the AD variables.
While we make the optimization to allocate these variables only one time for every library before the benchmark begins,
we cannot optimize-out the initialization
at every iteration to save the cost of copying.

The first few subsections will cover some micro-benchmarks where we benchmark commonly used functions: 
summation and product of elements in a vector, 
log-sum-exponential, 
matrix multiplication, 
and normal log-probability density function (pdf).
The last two subsections will cover some macro-benchmarks where we benchmark real-world examples: 
a regression model and a stochastic volatility model.

\subsection{Sum and Product}

\begin{figure*}[t]
    \centering
    \begin{subfigure}[b]{0.475\textwidth}
        \centering
        \includegraphics[width=\textwidth]{figs/sum_fig.png}
        \caption{Sum}\label{fig:sum}
    \end{subfigure}
    \hfill
    \begin{subfigure}[b]{0.475\textwidth}
        \centering
        \includegraphics[width=\textwidth]{figs/sum_iter_fig.png}
        \caption{Sum (naive, for-loop-based)}\label{fig:sum_iter}
    \end{subfigure}
    \vskip\baselineskip%
    \begin{subfigure}[b]{0.475\textwidth}
        \centering
        \includegraphics[width=\textwidth]{figs/prod_fig.png}
        \caption{Product}\label{fig:prod}
    \end{subfigure}
    \hfill
    \begin{subfigure}[b]{0.475\textwidth}
        \centering
        \includegraphics[width=\textwidth]{figs/prod_iter_fig.png}
        \caption{Product (naive, for-loop-based)}\label{fig:prod_iter}
    \end{subfigure}
    \caption{%
        Sum and product benchmarks of other libraries against FastAD 
        plotted relative to FastAD average time.
        Fig.~\ref{fig:sum},\ref{fig:prod} use built-in functions whenever available.
        Fig.~\ref{fig:sum_iter},\ref{fig:prod_iter} use the naive iterative-based code for all libraries.
    }\label{fig:sum_prod}
\end{figure*}

The summation function is defined as $f_s:\R^n \to \R$ where~$f_s(x) = \sum\limits_{i=1}^n x_i$,
and the product function is defined as $f_p:\R^n \to \R$ where~$f_p(x) = \prod\limits_{i=1}^n x_i$.
The only libraries supporting a built-in function for summation are Adept, FastAD, and Stan;
those that support product are Adept and FastAD.\@
All other libraries use the \verb|Eigen| API:
\begin{lstlisting}[style=customcpp]
    template <class T>
    T operator()(const Eigen::Matrix<T, Eigen::Dynamic, 1>& x) const
    { return x.sum(); }
\end{lstlisting}
As another test, we forced all libraries to use the manually-written for-loop-based summation and product.
Fig.\ref{fig:sum_prod} shows the benchmark results.

In all four cases, it is clear that FastAD outperforms all libraries for all values of $N$
by a significant factor.
For \verb|sum|, Fig.\ref{fig:sum} shows that, asymptotically, 
the next three fastest libraries are Adept, Stan, and Sacado, respectively.
The trend stabilizes starting from $N=2^6=64$ where Adept is about $ 12$ times slower than
FastAD, Stan about $ 18$ times, and Sacado about $ 74$ times.
The naive, for-loop-based benchmark shown in Fig.\ref{fig:sum_iter} exhibits a similar pattern,
although the performance difference with FastAD is less extreme:
Adept is about $ 4$ times slower, 
Stan about $ 6$ times,
and Sacado about $ 6.5$ times.
Nonetheless, this is still a very notable difference.
Although the difference between FastAD and the \verb|double| baseline is much larger in
the \verb|sum| case (2.56 times slower than baseline) than 
\verb|sum_iter| (1.76 times slower than baseline), 
this does not mean that \verb|sum| was slower.
In fact, in terms of absolute time, \verb|sum| was $ 5$ times faster than \verb|sum_iter|. 
Rather this means that the backward-evaluation in the case for \verb|sum| was
as expensive as the forward-evaluation, but in the case for \verb|sum_iter|,
backward-evaluation was much cheaper relative to the forward-evaluation.
This makes sense because \verb|sum| is vectorized and hence takes much shorter time to evaluate
than \verb|sum_iter|, and in both cases the backward-evaluation simply sets the partial derivatives to $ 1$.

For the \verb|prod| and \verb|prod_iter| cases, 
Fig.\ref{fig:prod},\ref{fig:prod_iter} again show
a similar trend as \verb|sum| and \verb|sum_iter|.
Overall, the comparison with FastAD is less extreme.
For \verb|prod|, Adept is about $ 5.6$ times slower than FastAD,
Stan about $ 7.3$ times slower,
and Sacado about $ 16.4$ times slower.
For \verb|prod_iter|, Adept is about $2.9$ times slower,
Stan about $4$ times slower,
and Sacado about $4.6$ times slower.


\subsection{Log-Sum-Exp}

\begin{figure*}[t]
    \centering
    \includegraphics[width=\textwidth]{figs/log_sum_exp_fig.png}
    \caption{%
        Log-sum-exp benchmark of other libraries against FastAD 
        plotted relative to FastAD average time.
    }\label{fig:log_sum_exp}
\end{figure*}

The log-sum-exp function is defined as $f:\R^n \to \R$ 
where~$f(x) = \log(\sum\limits_{i=1}^n e^{x_i})$,
The only library supporting a built-in function for log-sum-exp is Stan.
Fig.\ref{fig:log_sum_exp} shows the benchmark results.

FastAD outperforms all libraries for all values of $N$.
The next three fastest libraries are Adept, Stan, and Sacado, respectively.
The trend stabilizes starting from $N=2^6=64$ where 
Adept is about $ 3$ times slower than FastAD, 
Stan about $ 5$ times, and 
Sacado about $ 17$ times.
Note that FastAD is only marginally slower than the baseline
despite calls to expensive functions like \code{log} and \code{exp}.
In fact, FastAD is only $ 1.5$ times slower than the baseline, 
meaning there is only about $ 50\%$
overhead from one forward-evaluation to also compute the gradient.
This is not surprising since FastAD reuses
the forward-evaluated result for \code{exp} expression
during the backward-evaluation (see Section~\ref{ssec:unary}).


\subsection{Matrix Multiplication}\label{ssec:matrix_mult}

\begin{figure*}[t]
    \centering
    \includegraphics[width=0.8\textwidth]{figs/matrix_product_fig.png}
    \caption{%
        Matrix multiplication benchmark of other libraries against FastAD 
        plotted relative to FastAD average time.
    }\label{fig:matrix_mult}
\end{figure*}

For this benchmark, all matrices are square matrices of the same size.
In order to have a scalar target function,
we add another step of adding all of the entries of the matrix multiplication, i.e.
\[
    f(A, B) = \sum\limits_{i=1}^{K} \sum\limits_{j=1}^{K} {(A \cdot B)}_{ij}
\]
Fig.~\ref{fig:matrix_mult} shows the benchmark results.

FastAD is still the fastest library for all values of $N$, 
but Stan performs much closer to FastAD than in the previous examples.
All other libraries consistently take longer than both FastAD and Stan as $N$ increases.
Towards the end at around $N=2^{14}$, 
Stan is about $ 2$ times slower.
For moderate sized $N \in [2^{8}, 2^{10}]$, Stan is about $ 3$ times slower.

This example really highlights the benefits of vectorization.
As noted in Section~\ref{ssec:vectorization},
this was the one benchmark example where Stan was able to produce vectorized code,
which is consistent with Figure~\ref{fig:matrix_mult} 
that Stan is the only library that has the same 
order of magnitude as FastAD.
Other libraries did not produce vectorized code.

The comparison with the baseline shows that FastAD takes $ 3.27$ times longer.
Note that forward-evaluation requires one matrix multiplication between two $K\times K$ matrices,
and backward-evaluation additionally requires two matrix multiplications of the same order,
one for each adjoint:
\begin{align*}
    \frac{\partial f}{\partial A} 
    &= \frac{\partial f}{\partial (A\cdot B)} \cdot B^T, \,
    \frac{\partial f}{\partial B} 
    = A^T \cdot \frac{\partial f}{\partial (A\cdot B)}
\end{align*}
Hence, in total, one AD evaluation requires three matrix multiplications between two $K\times K$ matrices.
If we approximate a manually-written gradient computation to take 
three times as long as the baseline (one multiplication), 
FastAD time relative to this approximated time
is $\frac{3.27}{3} = 1.09$.
This shows then that FastAD only has about $ 9\%$ overhead 
from a manually-written code, which is extremely optimal.


\subsection{Normal Log-PDF}

\begin{figure*}[t]
    \centering
    \includegraphics[width=\textwidth]{figs/normal_log_pdf_fig.png}
    \caption{%
        Normal log-pdf benchmark of other libraries against FastAD 
        plotted relative to FastAD average time.
    }\label{fig:normal_log_pdf}
\end{figure*}

The normal log-pdf function is defined up to a constant:
\[
    f(x) = -\frac{1}{2} \sum\limits_{i=1}^N \paren{\frac{x_i - \mu}{\sigma}}^2 
           -N\log(\sigma)
\]
For this benchmark, $\mu = -0.56,\,\sigma = 1.37$ and are kept as constants;
the values themselves are not so important so long as $\sigma > 0$.
Only Stan and FastAD provide built-in functions to compute this function.
All other libraries are written with the \verb|Eigen| API to compute the sum in $f$ except Adept.
Adept provides \verb|norm2| function and we attempted to try the following:
\begin{lstlisting}[style=customcpp]
    adept::aReal operator()(const adept::aVector& x) const
    {
        return -adept::norm2(x - mu_) / (2 * sigma_ * sigma_) 
                - (x.size() * std::log(sigma_));
    }
\end{lstlisting}
however, this does not produce the correct adjoints.
Hence, since Adept does not interface well with \verb|Eigen|~\cite{hogan:2014}, 
our only choice was to write a naive for-loop method.
Fig.\ref{fig:normal_log_pdf} shows the benchmark results.

FastAD is the fastest library for all values of $N$.
It is not surprising that Stan is the next fastest, 
since the Stan Math Library was written precisely for differentiating probability densities.
It is, however, surprising that Adept is the third fastest,
given that we wrote our own for-loop to implement its overload.
The trend stabilizes towards at around $N=2^{7}=128$,
and towards the end
Stan is about $ 22$ times slower,
Adept about $ 41$ times,
and Sacado about $ 198$ times.


\subsection{Bayesian Linear Regression}\label{ssec:regression}

\begin{figure*}[t]
    \centering
    \includegraphics[width=\textwidth]{figs/regression_fig.png}
    \caption{%
        Bayesian linear regression benchmark of Stan against FastAD 
        plotted relative to FastAD average time.
    }\label{fig:regression}
\end{figure*}

This section marks the first macro-benchmark example.
We consider the following Bayesian linear regression model:
\begin{align*}
    y &\sim N\paren{X\cdot w + b, \sigma^2} \\
    w &\sim N\paren{0,1} \\
    b &\sim N\paren{0,1} \\
    \sigma &\sim Unif\paren{0.1, 10.}
\end{align*}
The target function is the log of the joint probability density function (up to a constant).
For this benchmark, we only consider Stan since they specialize in differentiating such functions
and because the other libraries were not well-suited to implement this efficiently.
The fill function for this functor will resize a vector of size $N = 2^K$ as $\tilde{N} = (K + 1) \cdot 10 + 2$,
where the first $(K+1) \cdot 10$ values refer to $w$ and the rest refer to $b$ and $\sigma$, respectively.
It also resizes its private members $y \in \R^{1000}$ and $X \in \R^{1000 \times \tilde{N}}$, which are constants.
All quantities are randomly generated uniformly in $(-1,1)$ range,
but $\sigma$ modified to be strictly positive.

FastAD outperforms Stan by 8 times for the largest $N$.
The trend stabilizes starting from around $N=70$.
It is interesting to see that FastAD is only 2 times slower than the baseline,
despite the model consisting of a large matrix multiplication and many normal log-pdf functions.
One of the reasons is that the compiler was able to optimize-out backward-evaluation for $X$, 
since constants implement a no-op for backward-evaluation.
If we assume that the most expensive operation is the matrix multiplication,
AD evaluation approximately takes 2 matrix multiplication between a matrix and a vector.
We can then approximate a lower bound for manually-written gradient computation to be two times the baseline.
The relative time of FastAD to this approximation is
$1.018$, implying less than $ 1.8\%$ overhead from a manually-written code.


\subsection{Stochastic Volatility}\label{ssec:stochastic_volatility}

\begin{figure*}[t]
    \centering
    \includegraphics[width=\textwidth]{figs/stochastic_volatility_fig.png}
    \caption{%
        Stochastic volatility benchmark of Stan against FastAD 
        plotted relative to FastAD average time.
    }\label{fig:stochastic_volatility}
\end{figure*}

This section marks the second and last macro-benchmark example.
We consider the following stochastic volatility model 
taken from \todo{how to cite Stan documentation?}:
\begin{align*}
    y &\sim N(0, e^{h}) \\
    h_{std} &\sim N(0, 1) \\
    \sigma &\sim Cauchy(0,5) \\
    \mu &\sim Cauchy(0,10) \\
    \phi &\sim Unif(-1, 1) \\
    h &= h_{std} \cdot \sigma \\
    h[0] &= \frac{h[0]}{\sqrt{1 - \phi^2}} \\
    h &= h + \mu \\
    h[i] &= \phi \cdot (h[i-1] - \mu),\, i > 0
\end{align*}
The target function is the log of the joint probability density function (up to a constant)
and we wish to differentiate it with respect to $h_{std}, h, \phi, \sigma, \mu$.
For this benchmark, we only consider Stan for the same reasons described in Section~\ref{ssec:regression}.
The fill function for this functor will resize a vector of size $N$ as $\tilde{N} = N + 3$,
where the first $N/2$ values refer to $h_{std}$,
the next $N/2$ values refer to $h$,
and the next three refer to $\phi, \sigma, \mu$, respectively.
We take care of the edge case where $N < 2$ to let $N = 2$ and then carry out the steps above.
Here, $y$ is a constant.
All quantities are random generated uniformly in $(-1,1)$ range,
but $\sigma$ is replaced with its absolute value plus $0.1$ to make it strictly positive.

The following is the Stan functor overload:
\begin{lstlisting}[style=customcpp]
    using namespace stan::math;
    using vec_t = Eigen::Matrix<var, Eigen::Dynamic, 1>;
    size_t N = (x.size() - 3) / 2;
    Eigen::Map<vec_t> h_std(x.data(), N);
    Eigen::Map<vec_t> h(x.data() + N, N);
    auto& phi = x(2*N);
    auto& sigma = x(2*N + 1);
    auto& mu = x(2*N + 2);
    h = h_std * sigma;
    h[0] /= sqrt(1. - phi * phi);
    h += mu * Eigen::VectorXd::Ones(N);
    for (size_t i = 1; i < N; ++i) {
        h[i] += phi * (h[i-1] - mu);
    }
    auto lp = normal_lpdf(y, 0., exp(h / 2.)) +
            normal_lpdf(h_std, 0., 1.) +
            cauchy_lpdf(sigma, 0., 5.) +
            cauchy_lpdf(mu, 0., 10.) +
            uniform_lpdf(phi, -1., 1.);
\end{lstlisting}
The following is the FastAD functor overload:
\begin{lstlisting}[style=customcpp]
    size_t N = (x.size() - 3) / 2;
    ad::VarView<T, ad::vec> h_std(x.data(), x.data_adj(), N);
    ad::VarView<T, ad::vec> h(x.data() + N, x.data_adj() + N, N);
    ad::VarView<T> phi(x.data() + 2*N, x.data_adj() + 2*N);
    ad::VarView<T> sigma(x.data() + 2*N + 1, x.data_adj() + 2*N + 1);
    ad::VarView<T> mu(x.data() + 2*N + 2, x.data_adj() + 2*N + 2);
    auto tp = (
        h = h_std * sigma,
        h[0] /= ad::sqrt(1. - phi * phi),
        h += mu,
        ad::for_each(counting_iterator<>(1),
                     counting_iterator<>(N),
                     [&](size_t i) { return h[i] += phi * (h[i-1] - mu); })
    );
    auto lp = ad::normal_adj_log_pdf(y, 0., ad::exp(h / 2.)) 
            + ad::normal_adj_log_pdf(h_std, 0., 1.)
            + ad::cauchy_adj_log_pdf(sigma, 0., 5.)
            + ad::cauchy_adj_log_pdf(mu, 0., 10.) 
            + ad::uniform_adj_log_pdf(phi, -1., 1.);
    return (tp, lp);
\end{lstlisting}
Fig.\ref{fig:stochastic_volatility} shows the benchmark results.

FastAD outperforms Stan by 3 times for the largest $N$.
The trend seems quite stabilized from the start.
It is interesting to see that FastAD is only marginally slower than the \verb|double| baseline
for moderate to large $N$ values.
The ratio between FastAD time to baseline time is $ 1.07$, which means there is about a $7\%$ overhead only
from just one forward-evaluation to also compute the gradient.
Hence, gradient computation is almost free from just computing the log-pdf,
which puts FastAD at near-optimal performance.

